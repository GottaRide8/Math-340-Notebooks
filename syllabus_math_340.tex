 \documentclass[a4paper,11pt]{article}

\usepackage{amsmath}
\usepackage{amssymb}
\usepackage{amsthm}
\usepackage{graphicx}
\usepackage{caption}
%\usepackage{subcaption}

\newtheorem{thm}{Theorem}
\newtheorem{lem}{Lemma}

\newcommand{\beq}{\begin{equation}}
\newcommand{\eeq}{\end{equation}}

\newcommand{\ba}{\begin{array}}
\newcommand{\ea}{\end{array}}

\newcommand{\bea}{\begin{eqnarray}}
\newcommand{\eea}{\end{eqnarray}}

\newcommand{\bc}{\begin{center}}
\newcommand{\ec}{\end{center}}

\newcommand{\ds}{\displaystyle}

\newcommand{\bt}{\begin{tabular}}
\newcommand{\et}{\end{tabular}}

\newcommand{\bi}{\begin{itemize}}
\newcommand{\ei}{\end{itemize}}

\newcommand{\bd}{\begin{description}}
\newcommand{\ed}{\end{description}}

\newcommand{\bp}{\begin{pmatrix}}
\newcommand{\ep}{\end{pmatrix}}

\newcommand{\p}{\partial}
\newcommand{\sech}{\mbox{sech}}
\newcommand{\cf}{{\it cf.}~}
\title{Math 340: Programming in Mathematics}
\date{}
\begin{document}
\maketitle

\noindent{\bf Course Information}:\\
\\
\begin{tabular}[]{ll}
Instructor:& Chris Curtis\\
Office:& GMCS 591\\
Meeting Times: & M/W, 15:30-16:45 in GMCS 421\\
Office Hours:& T, 9-12, in GMCS 591\\
Email: & ccurtis@sdsu.edu\\
\end{tabular}
\\ \\
\noindent {\bf Prerequisites}: Mathematics 150,151,245  \\
\\
{\bf Official Course Description}: Introduction to Python programming. Modeling, problem solving, and visualization.\\
\\
{\bf Learning Outcomes}:
The overaching outcomes in this course will be for students to, using the Python programming language, 
\begin{enumerate}
\item Procedural Programming in Python: Students will define and use: data types, conditional statements, while and for loops, functions.
\begin{itemize}
\item Course Activity:  This will be done through active learning practices done during lecture.  
\item Assessment:  Student facility with this outcome will be assessed through performance on assignments and exams.  
\end{itemize}
\item Basic Data Structures: For storing and manipulating data, students will use: NumPy arrays, lists, tuples, dictionaries.
\begin{itemize}
\item Course Activity:  This will be done through active learning practices done during lecture.  
\item Assessment: Student facility with this outcome will be assessed through performance on assignments and exams.  
\end{itemize}
\item Doing Mathematics on Computers: Students will practice translating problems and notation in mathematics into algorithms via the Python language.
\begin{itemize}
\item Course Activity:  This will be done through active learning practices done during lecture.  
\item Assessment: Student facility with this outcome will be assessed through performance on assignments and exams.  
\end{itemize}  
\item Data Visualization and Manipulation: Students will learn how to read and write to files.
\begin{itemize}
\item Course Activity:  This will be done through active learning practices done during lecture.  
\item Assessment: Student facility with this outcome will be assessed through performance on assignments and exams.  
\end{itemize}  
\item Working in Modern Computing Environments: Students will use the features of various programming environments to solve problems and present their work.
\begin{itemize}
\item Course Activity: Active learning practices done during lecture.
\item Assessment: Student facility with this outcome will be assessed through performance on assignments and exams.
\end{itemize}  
\end{enumerate}   

\noindent {\bf Grading Policy}: Your final score will consist of homework (40\%), one midterm (30\%), and a final exam (30\%).  Homework is roughly due every  week, though please pay attention to the schedule since there are exceptions to this (and every) rule.  \\
\\
{\bf Homework Policy}:  Work you submit should be as stand alone Python files or in Jupyter notebooks per the request of the problem.  Late work is not accepted unless you make arrangements with me in advance.\\  
\\
{\bf Students with Disabilities}: If you are a student with a disability and believe you will need accommodations for this class, it is your responsibility to contact Student Disability Services at (619) 594-6473. To avoid any delay in the receipt of your accommodations, you should contact Student Disability Services as soon as possible. Please note that accommodations are not retroactive, and that accommodations based upon disability cannot be provided until you have presented your instructor with an accommodation letter from Student Disability Services. Your cooperation is appreciated.\\
\pagebreak
\begin{center}
\begin{tabular}[]{cc|l}
Week & Dates & Sections \\
\hline
Week 1 & 08/26, 08/28 & \\
Week 2 & 09/02, 09/04 & No Class\\
Week 3 & 09/09, 09/11 & \\
Week 4 & 09/16, 09/18 & \\
Week 5 & 09/23, 09/25 & \\
Week 6 & 09/30, 10/02 & \\
Week 7 & 10/07, 10/09 & Midterm I (10/09)\\
Week 8 & 10/14, 10/16 & \\
Week 9 & 10/21, 10/23 & \\
Week 10 & 10/28, 10/30 & \\
Week 11 & 11/04, 11/06 & No Class (11/04)\\
Week 12 & 11/11, 11/13 & \\
Week 13 & 11/18, 11/20 & \\
Week 14 & 11/25, 11/27 & No Class(11/27)\\
Week 15 & 12/02, 12/04 & \\
Week 16 & 12/09, 12/11 & Last Day of Class(12/11)\\
\end{tabular}
\end{center}


\end{document}